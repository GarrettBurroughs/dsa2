\documentclass[11pt]{article}
%\usepackage{fancyheadings}
\usepackage{wrapfig}
\usepackage{epsfig}
\usepackage{hyperref}
\setlength{\headheight}{0pt}
%\setlength{\footheight}{0pt}
\setlength{\topmargin}{-.5in}
\setlength{\oddsidemargin}{-0.25in}
\setlength{\evensidemargin}{-0.25in}
\setlength{\textwidth}{7truein}
\setlength{\textheight}{9truein}
\setlength{\parskip}{6pt}

\begin{document}

\section*{Train Stations}

%\subsection*{Description}

\begin{wrapfigure}{r}{3in}
\vspace{-10pt}
\epsfig{figure=train,width=3in}
\vspace{-30pt}
\end{wrapfigure}

You have decided to create a train system for your city, but you need to minimize the cost of construction. Given the list of train stations and the cost to build the potential rails between them. Design a minimum cost train system.

Your solution must consider the following:

\begin{itemize}
	\item All train stations must be accessible from all other train stations.
	\item There can be no cycles in the network (i.e., the network must be a tree to keep costs down)
	\item Your solution MUST implement Kruskal's algorithm
	\item You MUST implement your own custom find-union data structure.
\end{itemize}

\subsection*{Input}
The input file will begin with an integer $n$, the number of train stations (indexed from $0$ to $n-1$) and $e$, the number of potential rails between train stations. The next $e$ lines will list out three numbers, $s$, $t$, and $c$ representing two stations ($s$ and $t$) between which a rail can be built and the cost $c$ of building such a rail.

\subsection*{Output}

Output the minimum cost of building the train network.

\vspace{0.25in}\hspace{-0.3in}\begin{tabular}{ll}

%\subsection*{Sample Input}
\parbox{3in}{{\large\bf Sample Input}

\vspace{0.15in}

{\tt 
5 6\linebreak
0 1 1\linebreak
0 3 1\linebreak
1 3 2\linebreak
1 2 2\linebreak
2 3 1\linebreak
3 4 7
}
}

&

\parbox{3in}{{\large\bf Sample Output}

\vspace{0.15in}

{\tt
10
}
}

\\
\end{tabular}

\end{document}
