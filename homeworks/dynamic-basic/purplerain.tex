\documentclass[11pt]{article}
%\usepackage{fancyheadings}
\usepackage{wrapfig}
\usepackage{epsfig}
\setlength{\headheight}{0pt}
%\setlength{\footheight}{0pt}
\setlength{\topmargin}{-.5in}
\setlength{\oddsidemargin}{-0.25in}
\setlength{\evensidemargin}{-0.25in}
\setlength{\textwidth}{7truein}
\setlength{\textheight}{9truein}
\setlength{\parskip}{6pt}

\begin{document}

\section*{Purple Rain}
%\subsection*{Description}

\begin{wrapfigure}{r}{3in}
\vspace{-10pt}
\epsfig{figure=greenrain,width=3in}
\vspace{-30pt}
\end{wrapfigure}

Sometimes (during the summer when students are away), purple rain falls in the magical kingdom of the engineering school, which is a straight and thin row walkway surrounded by engineering buildings.

On close observation however, Prof. Floryan finds that actually
it is a mix of red and blue drops.

In his zeal, he records the location and color of the raindrops in
different locations along the e-school.
Looking at the data, Professor Floryan wants to know which part of
the e-school had the ``least'' purple rain.

After some thought, he decides to model this problem as follows.
Divide the e-school into $n$ sections and number them West to East
from $1$ to $n$.  Then, describe the raindrops as a sequence of {\tt
R} and {\tt B}, depending on whether the rainfall in each section is
primarily red or blue.  Finally, find a subsequence of where the
difference between the number of {\tt R} and the number of {\tt B} is
maximized.

\subsection*{Input}

The input consists of a single line containing a string of $n$
characters ($1 \le n \le 10^5$), describing the color of the raindrops
in sections $1$ to $n$.

It is guaranteed that the string consists of uppercase ASCII letters
`{\tt R}' and `{\tt B}' only.

\subsection*{Output}

Print, on a single line, two space-separated integers that describe
the starting and ending positions of the part of the e-school that had
the least purple rain.

If there are multiple possible answers, print the one that has the
Westernmost (smallest-numbered) starting section.  If there are multiple
answers with the same Westernmost starting section, print the one with
the Westernmost ending section.


\vspace{0.25in}\hspace{-0.3in}\begin{tabular}{ll}

%\subsection*{Sample Input}
\parbox{3in}{{\large\bf Sample Input}

\vspace{0.15in}

{\tt 
//example 1\linebreak
BBRRBRRBRB\linebreak
\linebreak
//example 2\linebreak
BBRBBRRB
}
}

&

\parbox{3in}{{\large\bf Sample Output}

\vspace{0.15in}

{\tt
3 7\linebreak
\linebreak
1 5
}


}

\\
\end{tabular}

\end{document}
