\documentclass[12pt]{article}
\usepackage[top=1in,bottom=1in,left=0.75in,right=0.75in,centering]{geometry}
\usepackage{fancyhdr}
\usepackage{epsfig}
\usepackage[pdfborder={0 0 0}]{hyperref}
\usepackage{palatino}
\usepackage{wrapfig}
\usepackage{lastpage}
\usepackage{color}
\usepackage{ifthen}
\usepackage[table]{xcolor}
\usepackage{graphicx,type1cm,eso-pic,color}
\usepackage{hyperref}
\usepackage{amsmath}
\usepackage{wasysym}

\def\course{CS 4102: Algorithms}
\def\homework{M2: Divide and Conquer - Recurrence Relations}
\def\semester{Fall 2021}

\newboolean{solution}
\setboolean{solution}{false}

% add watermark if it's a solution exam
% see http://jeanmartina.blogspot.com/2008/07/latex-goodie-how-to-watermark-things-in.html
\makeatletter
\AddToShipoutPicture{%
\setlength{\@tempdimb}{.5\paperwidth}%
\setlength{\@tempdimc}{.5\paperheight}%
\setlength{\unitlength}{1pt}%
\put(\strip@pt\@tempdimb,\strip@pt\@tempdimc){%
\ifthenelse{\boolean{solution}}{
\makebox(0,0){\rotatebox{45}{\textcolor[gray]{0.95}%
{\fontsize{5cm}{3cm}\selectfont{\textsf{Solution}}}}}%
}{}
}}
\makeatother

\pagestyle{fancy}

\fancyhf{}
\lhead{\course}
\chead{Page \thepage\ of \pageref{LastPage}}
\rhead{\semester}
%\cfoot{\Large (the bubble footer is automatically inserted into this space)}

\setlength{\headheight}{14.5pt}

\newenvironment{itemlist}{
\begin{itemize}
\setlength{\itemsep}{0pt}
\setlength{\parskip}{0pt}}
{\end{itemize}}

\newenvironment{numlist}{
\begin{enumerate}
\setlength{\itemsep}{0pt}
\setlength{\parskip}{0pt}}
{\end{enumerate}}

\newcounter{pagenum}
\setcounter{pagenum}{1}
\newcommand{\pageheader}[1]{
\clearpage\vspace*{-0.4in}\noindent{\large\bf{Page \arabic{pagenum}: {#1}}}
\addtocounter{pagenum}{1}
\cfoot{}
}

\newcounter{quesnum}
\setcounter{quesnum}{1}
\newcommand{\question}[2][??]{
\begin{list}{\labelitemi}{\leftmargin=2em}
\item [\arabic{quesnum}.] {} {#2}
\end{list}
\addtocounter{quesnum}{1}
}


\definecolor{red}{rgb}{1.0,0.0,0.0}
\newcommand{\answer}[2][??]{
\ifthenelse{\boolean{solution}}{
\color{red} #2 \color{black}}
{\vspace*{#1}}
}

\definecolor{blue}{rgb}{0.0,0.0,1.0}

\begin{document}

\section*{\homework}

This homework will ask you to think about Divide and Conquer Algorithms with a focus on recurrence relations (this is module 2 after all!). Please try to follow the guidelines below when writing up your solutions to these problems:

\begin{itemize}
	\item Algorithm descriptions should be about 1 paragraph long. If you are writing more than one paragraph to describe your algorithm, then it is TOO long. 
	\item There should be enough detail that I could implement the algorithm if I wanted to, but we don't need to see code. If you want to add pseudo-code for clarity, that would be fine.
	\item For the recurrence relation problems below, make sure to show your work clearly!
\end{itemize}

\vspace{14pt}

\question[3]{
You are a hacker, trying to gain information on a secret array of size $n$. This array contains $n-1$ ones and exactly $1$ two; you want to determine the index of the two in the array.\\
\\
Unfortunately, you don't have access to the array directly; instead, you have access to a function $f(l1, l2)$ that compares the sum of the elements of the secret array whose indices are in $l1$ to those in $l2$. This function returns $-1$ if the $l1$ sum is larger, $0$ if they are equal, and $1$ if the sum corresponding to $l2$ is larger.\\
\\
For example, if the array is $a=[1,1,1,2,1,1]$ and you call $f([0,2,4],[1,3,5])$ then the return value is $1$ because $a[0]+a[2]+a[4]=3<4=a[1]+a[3]+a[5]$. Design an algorithm to find the index of the $2$ in the array using the least number of calls to $f()$. Then, answer the following questions:

\begin{itemize}
	\item Describe your algorithm clearly (in a paragraph or so)
	\item Give the recurrence relation for the runtime of your algorithm. Make sure to write this in terms of $f_r$, which we will use to represent the runtime of $f()$.
	\item Suppose you discover that $f()$ runs in $\Theta(max(|l1|,|l2|))$, what is the overall runtime of your algorithm in this case? 
\end{itemize}
}

\vspace{12pt}


%----------------------------------------------------------------------
\noindent Directly solve, by unrolling the recurrence, the following relation to find its exact solution. Make sure to show your work.

\question[2]{
$T(n) = T(n-1) + n$
}

\vspace{12pt}

%----------------------------------------------------------------------

\noindent Use induction to show bounds on the following recurrence relations.

\question[2]{
Show that $T(n)=4T(\frac{n}{3})+n \in O(n^{log_3(4)})$. You'll need to subtract off a lower-order term to make the induction work here.
}

\answer[0 in]{
...
}

\vspace{12pt}

%----------------------------------------------------------------------

\noindent Use the master theorem to solve the following recurrence relations. State which case of the theorem you are using and why.

\question[2]{
$T(n)=2T(\frac{n}{4})+1$
}

\answer[0 in]{
...
}

\question[2]{
$T(n)=2T(\frac{n}{4})+\sqrt{n}$
}

\answer[0 in]{
...
}

\question[2]{
$T(n)=2T(\frac{n}{4})+n$
}

\answer[0 in]{
...
}

\question[2]{
$T(n)=2T(\frac{n}{4})+n^2$
}

\answer[0 in]{
...
}
\end{document}
